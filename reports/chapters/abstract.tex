%Two resources useful for abstract writing.
% Guidance of how to write an abstract/summary provided by Nature: https://cbs.umn.edu/sites/cbs.umn.edu/files/public/downloads/Annotated_Nature_abstract.pdf %https://writingcenter.gmu.edu/guides/writing-an-abstract
\chapter*{\center \Large  Abstract}
%%%%%%%%%%%%%%%%%%%%%%%%%%%%%%%%%%%%%%
% Replace all text with your text
%%%%%%%%%%%%%%%%%%%%%%%%%%%%%%%%%%%

 Credit card fraud has been a major concern for all financial institutions and 
 banking sectors and to avoid it, strict security and safety measures are 
 required. Due to an increase in digital transactions, a need for an efficient 
 fraud detection system is very important as a significant rise in cybecrime 
 has been observed. The main purpose of this report is to develop an efficient 
 fraud detection system using various machine learning models to increase the 
 precision and effectiveness of detecting any fraud in online credit card 
 transactions by obtaining the past details of transactions of the customers 
 and looking for patterns to detect any suspicious activity. A credit card 
 fraud detection dataset is used which is taken from an online source. This 
 dataset contains all the necessary information which is required by the 
 models to train and detect fraud easily. The first step is to clean and 
 process the data to extract the relevant information which is then used to 
 train the various machine learning models to detect fraud. Various machine 
 learning models used are Logistic Regression, Decision Tree Classification, 
 and Random forest classification which train on the processed data and are 
 efficient in giving accurate results. These models are evaluated based on 
 certain performance metrics like precision, F1 score, and recall and also 
 guarantee the best possible balance between false positives and false 
 negatives,  which can be assessed to provide the financial sectors with a 
 more efficient approach to detect and avoid fraud activities while doing 
 online transactions.

%%%
~\\[1cm]%REMOVE THIS



%%%%%%%%%%%%%%%%%%%%%%%%%%%%%%%%%%%%%%%%%%%%%%%%%%%%%%%%%%%%%%%%%%%%%%%%%s
~\\[1cm]
\noindent % Provide your key words
\textbf{Keywords:} Logistic Regression, Decision Tree Classification, Random forest classification, F1 Score, False Positives, False Negatives

\vfill
\noindent


