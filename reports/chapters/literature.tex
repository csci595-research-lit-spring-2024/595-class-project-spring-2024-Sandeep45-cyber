\chapter{Literature Review}
\label{ch:lit_rev} %Label of the chapter lit rev. The key ``ch:lit_rev'' can be used with command \ref{ch:lit_rev} to refer this Chapter.

Credit card fraud causes significant financial losses for both people and businesses. This can have a detrimental effect on one's physical and emotional well-being. Furthermore, the aggregate impact of credit card theft raises interest rates and expenses for all clients as these financial institutions try to offset the losses caused by fraudulent activity. Generally speaking, credit card fraud has a lot of bad effects, which emphasizes the need for ongoing efforts to strengthen cybersecurity defenses and protect individuals and businesses from such monetary loss. Credit card fraud is extremely difficult to detect and has been more common recently. This
is because credit card information, such as the credit card number and expiration date, may be used by anybody to make transactions on websites without the owner's permission. First of all, victims occasionally find that opposing such illegal activity, navigating bureaucracy, and trying to retrieve their compromised money accounts are extremely taxing and frustrating procedures. By automatically learning the transaction data represented in a hierarchical manner, these deep
learning architectures have proven to perform better than other approaches, allowing them to identify complex patterns that may be missed by traditional techniques. In order to compare alternative approaches and improve model accuracy, the primary goal of this research is to evaluate the effectiveness of the user's fraud detection model using a variety of supervised machine algorithms, deep learning models, and artificial neural networks. An effective way to improve the working and enhance the performance of the implemented models is to evaluate
and analyze the model based on various evaluation metrics and performance metrics like F1 score, False positive and False negative, precision, recall, and many more.

In addition to securing bank and credit card profits and job security, these technologies will prevent fraud from costing financial institutions billions of dollars each year. Researchers have
also looked at integrating cutting-edge methods like reinforcement learning and natural
language processing (NLP) to further enhance fraud detection systems. NLP-based techniques
use textual data, including relevant information about the cardholder like the cardholder’s name,
past transactions, and the date and time of the transaction to determine useful characteristics
for fraud detection, while reinforcement learning algorithms are constantly optimizing their fraud
detection tactics in response to environmental input. Additionally, by finding relevant features
and lowering dimensional aspects, feature engineering, and selection strategies are important in
enhancing the different capabilities of various fraud detection systems. The importance of data pretreatment techniques like normalization, outlier identification, and imputations has also been
highlighted by researchers as a way to improve the accuracy and dependability of the dataset that has been provided to the machine learning model as an input.

\section{Existing System}

Credit card fraud detection systems are very significant in the field of research of academics
and economics. Several financial and other related institutions are willing to spend large
amounts of money in order to develop an efficient system to secure the process of online
transactions using credit cards for customers. Many different researchers from all around the
world have developed various systems using different machine learning techniques to identify
and prevent online financial fraud which is increasing every day and which can impose great
losses on users and specific financial institutions. Present-day systems on credit card fraud
detection make use of various machine learning and deep learning algorithms to build artificial
neural models and different optimization techniques in order to gain the desired outcome. An
existing system \citep{Priscilla-2020} that can efficiently stop online transactions using credit cards is based on the
approach of feature selection and other different machine learning algorithms. It adopts different
boosting classifiers like the XGBoosting algorithm, light gradient boosting algorithm (LGBM),
and Classic Gradient Boosting (CatBoost) to inspect the classification performance of the
system.

% PLEAE CHANGE THE TITLE of this section
\section{Example of in-text citation of references in \LaTeX} 
% Note the use of \cite{} and \citep{}
The references in a report relate your content with the relevant sources, papers, and the works of others. To include references in a report, we \textit{cite} them in the texts. In MS-Word, EndNote, or MS-Word references, or plain text as a list can be used. Similarly, in \LaTeX, you can use the ``thebibliography'' environment, which is similar to the plain text as a list arrangement like the MS word. However, In \LaTeX, the most convenient way is to use the BibTex, which takes the references in a particular format [see references.bib file of this template] and lists them in style [APA, Harvard, etc.] as we want with the help of proper packages.    

These are the examples of how to \textit{cite} external sources, seminal works, and research papers. In \LaTeX, if you use ``\textbf{BibTex}'' you do not have to worry much since the proper use of a bibliographystyle package like ``agsm for the Harvard style'' and little rectification of the content in a BiBText source file [In this template, BibTex are stored in the ``references.bib'' file], we can conveniently generate  a reference style. 

Take a note of the commands \textbackslash cite\{\} and \textbackslash citep\{\}. The command \textbackslash cite\{\} will write like ``Author et al. (2019)'' style for Harvard, APA and Chicago style. The command \textbackslash citep\{\} will write like ``(Author et al., 2019).'' Depending on how you construct a sentence, you need to use them smartly. Check the examples of \textbf{in-text citation} of sources listed here [This template recommends the \textbf{Harvard style} of referencing.]:
\begin{itemize}
    \item \cite{lamport1994latex} has written a comprehensive guide on writing in \LaTeX ~[Example of \textbackslash cite\{\} ].
    \item If \LaTeX~is used efficiently and effectively, it helps in writing a very high-quality project report~\citep{lamport1994latex} ~[Example of \textbackslash citep\{\} ].   
    \item A detailed APA, Harvard, and Chicago referencing style guide are available in~\citep{uor_refernce_style}.
\end{itemize}

\noindent 
Example of a numbered list:
\begin{enumerate}
    \item \cite{lamport1994latex} has written a comprehensive guide on writing in \LaTeX.
    \item If \LaTeX is used efficiently and effectively, it helps in writing a very high-quality project report~\citep{lamport1994latex}.   
\end{enumerate}

% PLEAE CHANGE THE TITLE of this section
\section{Example of ``risk'' of unintentional plagiarism}
Using other sources, ideas, and material always bring with it a risk of unintentional plagiarism. 

\noindent
\textbf{\color{red}MUST}: do read the university guidelines on the definition of plagiarism as well as the guidelines on how to avoid plagiarism~\citep{uor_plagiarism}.




% A possible section of you chapter
\section{Critique of the review} % Use this section title or choose a betterone
Describe your main findings and evaluation of the literature. ~\\

% Pleae use this section
\section{Summary} 
Write a summary of this chapter~\\
