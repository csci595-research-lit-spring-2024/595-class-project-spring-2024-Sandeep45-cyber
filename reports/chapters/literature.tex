\chapter{Literature Review}
\label{ch:lit_rev} %Label of the chapter lit rev. The key ``ch:lit_rev'' can be used with command \ref{ch:lit_rev} to refer this Chapter.

Credit card fraud causes significant financial losses for both people and businesses. This can have a detrimental effect on one's physical and emotional well-being. Furthermore, the aggregate impact of credit card theft raises interest rates and expenses for all clients as these financial institutions try to offset the losses caused by fraudulent activity. Generally speaking, credit card fraud has a lot of bad effects, which emphasizes the need for ongoing efforts to strengthen cybersecurity defenses and protect individuals and businesses from such monetary loss. Credit card fraud is extremely difficult to detect and has been more common recently. This
is because credit card information, such as the credit card number and expiration date, may be used by anybody to make transactions on websites without the owner's permission. First of all, victims occasionally find that opposing such illegal activity, navigating bureaucracy, and trying to retrieve their compromised money accounts are extremely taxing and frustrating procedures. By automatically learning the transaction data represented in a hierarchical manner, these deep
learning architectures have proven to perform better than other approaches, allowing them to identify complex patterns that may be missed by traditional techniques. In order to compare alternative approaches and improve model accuracy, the primary goal of this research is to evaluate the effectiveness of the user's fraud detection model using a variety of supervised machine algorithms, deep learning models, and artificial neural networks. An effective way to improve the working and enhance the performance of the implemented models is to evaluate
and analyze the model based on various evaluation metrics and performance metrics like F1 score, False positive and False negative, precision, recall, and many more.

In addition to securing bank and credit card profits and job security, these technologies will prevent fraud from costing financial institutions billions of dollars each year. Researchers have
also looked at integrating cutting-edge methods like reinforcement learning and natural
language processing (NLP) to further enhance fraud detection systems. NLP-based techniques
use textual data, including relevant information about the cardholder like the cardholder’s name,
past transactions, and the date and time of the transaction to determine useful characteristics
for fraud detection, while reinforcement learning algorithms are constantly optimizing their fraud
detection tactics in response to environmental input. Additionally, by finding relevant features
and lowering dimensional aspects, feature engineering, and selection strategies are important in
enhancing the different capabilities of various fraud detection systems. The importance of data pretreatment techniques like normalization, outlier identification, and imputations has also been
highlighted by researchers as a way to improve the accuracy and dependability of the dataset that has been provided to the machine learning model as an input.

\section{Existing System}

Credit card fraud detection systems are very significant in the field of research of academics
and economics. Several financial and other related institutions are willing to spend large
amounts of money in order to develop an efficient system to secure the process of online
transactions using credit cards for customers. Many different researchers from all around the
world have developed various systems using different machine learning techniques to identify
and prevent online financial fraud which is increasing every day and which can impose great
losses on users and specific financial institutions. Present-day systems on credit card fraud
detection make use of various machine learning and deep learning algorithms to build artificial
neural models and different optimization techniques in order to gain the desired outcome. An
existing system \citep{Priscilla-2020} that can efficiently stop online transactions using credit cards is based on the
approach of feature selection and other different machine learning algorithms. It adopts different
boosting classifiers like the XGBoosting algorithm, light gradient boosting algorithm (LGBM),
and Classic Gradient Boosting (CatBoost) to inspect the classification performance of the
system.

\section{Related Work}

Various researchers have proposed different models with different specifications and algorithms
with an aim of building a more efficient and promising fraud detection system. This research has
been an advancement in the field of academic and financial explorations. A system developed
by an author \citep{Asha-2021}  implements the use of various machine learning algorithms and artificial neural
networks such as Random Forest, K-means, and Support Vector Machine (SVM) to predict the
chances of fraud happening in a particular credit card transaction. In one of the research
papers, an author proposed an effective method that can easily detect any fraudulent activity in
online transactions which was carried out in four stages \citep{Jiang-2018}. Each stage had its significant
contribution in detecting the fraud. It uses different machine learning classifiers in order to train
each group where the customers who use credit cards are categorized into different groups

based on the data of their past transactions which helps the model to analyze the results in a
more informative manner. This model proved to be very efficient in its result as it also contained
a feedback mechanism for any new transactions made by the credit card holder. Speaking
about different machine learning models that already exist in the present day, another author
proposed a credit card fraud detection system \citep{Askari-2017}  that uses a machine learning algorithm to
detect online fraud transactions. This system used a machine learning model like Decision Tree
which used fuzzy logic and a decision tree classifier in its implementation. This system provided
very efficient and accurate results in comparison to other existing models. Another research
work by an author proposed a credit card fraud detection system that classifies the transactions
made by using a credit card based on a technique known as anomaly detection \citep{Pourhabibi-2020} which is
based on graphs. It implements the Graph-based Anomaly Detection (GBAD) technique which
is a popular method to analyze suspicious behavior and patterns and also determine the main
cause which helps in boosting the model’s performance for better outcomes. In addition, the
research work by an author used an alternative machine learning approach which is long-short-
term memory (LSTM) \citep{Jurgovsky-2018}  in order to detect fraud activities and prevent them using various
measures suggested by the relevant institutions who are responsible. This model was evaluated
based on a classifier that is Random Forest classifier and then the accuracy score was
calculated and evaluated as a performance metric of the working of the model. This method
used in the model is also known as the sequence classifier technique.

\section{Proposed System}

Financial organizations and customers throughout the world are very concerned about credit card fraud. Conventional rule-based fraud detection systems are losing their effectiveness due
to the growing complexity of criminals. As such, there's been an increasing amount of interest in
using machine learning algorithms to identify suspicious transactions. With an emphasis on the
use of machine learning approaches for credit card fraud detection, this survey of the literature
looks at the proposed system in the subject. The use of a wide range of machine learning
techniques and models is reflected by various researchers all around the world. These models
have proved to be very reliable and useful to catch a fraud intervention while performing online transactions. Researchers have also looked into integrating anomaly detection algorithms to find
odd patterns that point out suspicious activities. Deep learning techniques that use neural
networks like convolutional neural network (CNN), artificial neural network (ANN), and recurrent
neural network (RNN), have been developed to build an advanced system that can easily detect
any malicious behavior. The body of research emphasizes how crucial it is to continuously
improve and modify different machine learning models in order to counter the always-changing
field of credit card theft.


In this research work, various models are combined to get more effective and useful outcomes.
Subsequently, out of various models like Support Vector Machine(SVM), K-means
Classification, Decision Tree classifiers, and Regression-based classifiers, two classifiers are
combined which are a tree-based model and a liner-based model along with implementing a
grid-based searching algorithm. These models undergo optimization for the provided dataset by
the application of techniques such as hyperparameter tweaking, which substantially enhances
the model's performance in an optimized manner. In this way, the financial industry benefits
from a more effective method to identify and prevent fraudulent activity while conducting online
transactions thanks to this fraud detection technology.



In conclusion, the analysis of the literature about the suggested system of several references
for credit card fraud detection systems by using various machine learning techniques
demonstrates the wide array of techniques, methods, and strategies utilized in the fight against
online cheating and fraud. Researchers and scholars are continuously working on devising and
improving fraud detection systems to tackle the changing obstacles of such fraud related to
finance, ranging from conventional machine learning methods to deep learning frameworks. In
order to create a reliable and efficient fraud detection system that protects the institutions
related to finance and economy like banking sectors, businesses and consumer interests from
hostile activity, noble concerns, strong assessment frameworks, and ongoing observations are
some of the crucial components. In order to improve the safety and confidentiality of fraud
detection systems in this advanced digital and linked world, future research initiatives may
concentrate on using cutting-edge technologies like blockchain, advanced deep learning, etc.




