\chapter{Introduction}
\label{ch:into} % This how you label a chapter and the key (e.g., ch:into) will be used to refer this chapter ``Introduction'' later in the report. 
% the key ``ch:into'' can be used with command \ref{ch:intor} to refere this Chapter.

With the rise of digital transactions, the chances of online fraud have also increased 
significantly. Credit cards have become an integral part of financial transactions which facilitates online commerce more easily and it is also convenient for customers. However, with an increase in the use of such online transactions, there is an increase in the risk of fraud which leads to huge losses for both customers and financial institutions. Credit card fraud happens when an unidentified user makes the transaction using someone else’s card information.
Detecting such fraud manually is a very challenging task due to such a high number of transactions every moment. To deal with this cybercrime, an efficient fraud detection system \citep{Dornadula-2019} should be established which uses various machine learning algorithms and models to accurately detect the fraud. In this research, a credit card fraud detection dataset  is taken from an online source Kaggle \citep{Kaggle}  where the data is cleaned and processed to extract relevant information which can be used as training data for the machine learning models to accurately detect the fraud. Some machine learning algorithms \citep{Shah-2023} that are used are logistic regression,decision tree classification, and random forest classification which can detect fraud based on the data of past transactions and can detect fraud in real-time which is a big advantage of using machine learning algorithms. After that, these models are further tuned for provided dataset  using techniques like hyperparameter tuning \citep{Dalal-2022} , which greatly enhances the model's performance. The training data used in these models are based on various factors like location, date and time of the transaction, past transactions, etc. The performance of these combined models is then evaluated using various evaluation metrics. Deployment of such fraud detection systems ensures safer transactions by improving online security.




%%%%%%%%%%%%%%%%%%%%%%%%%%%%%%%%%%%%%%%%%%%%%%%%%%%%%%%%%%%%%%%%%%%%%%%%%%%%%%%%%%%
\section{Background}
\label{sec:into_back}

\subsection{Credit Card Fraud Definition}

Credit cards are frequently used for online shopping and banking, as well as other online activities. However, the growth and development of credit card use have also given rise to a variety of fraudulent activities. Fraud refers to the act of cheating someone for money which is illegal and people who commit fraud are called fraudsters. Fraud detection can be very challenging and it is important to track the fraudster to detect and prevent such cases due to which the innocent suffers. Therefore, a fraud detection system has been established with the help of modern tools and techniques like machine learning which has various deep learning and artificial neural models that can accurately predict and detect any unusual activity. Credit card fraud is a very common type of financial fraud these days in which an unauthorized person has access to someone’s credit card details and can make unidentified transactions which leads to severe loss for the credit card holders. It is very crucial to detect such unauthorized users because of which a detection system has to be established which makes the process easier with the help of machine learning techniques and various types of neural models 
\subsection{Credit Card Fraud Detection}

A modern detection system has been established that can easily identify and detect any fraud activities with the help of machine learning which provides various libraries and neural models that can easily predict such fraud. The model is provided with training data prepared by cleaning and analyzing the dataset containing the necessary information related to the credit card like the card holder's name, last transaction, date and time of transaction, amount of transaction, etc. Detecting credit card fraud at an early stage is very important to recover the financial loss suffered by the cardholder. Several other measures can be taken by the cardholder as soon as fraud is detected as they can immediately report the fraud at the specific bank, request the blocking of the credit card, etc. 
\subsection{Credit Card Fraud Identification}

Identifying credit card fraud is currently difficult because the majority of individuals are not aware of credit card theft, After all, the majority of scams emerge via legitimate channels that follow financial institutions and banks; the distinction is that these channels are unapproved third parties. It is very tricky and challenging to spot credit card fraud as it has recently increased because anybody who has credit card data which includes the credit card number, and expiration date can use it to make purchases on the website without their consent. In addition to having more chances to conduct fraudulent transactions by swiping credit cards—rather than only the ones we see—fraudsters will obtain more information about people's financial situations. 
\subsection{Credit Card Fraud Consequences}

Frauds related to credit cards impose huge financial losses on businesses and individuals. First of all, challenging such unlawful activities, negotiating bureaucracy, and attempting to recover their compromised financial accounts are sometimes very tiring and frustrating processes for victims. The impact on one's mental and physical health may be severe as a result. In addition, when these financial institutions attempt to make up for the losses brought on by fraudulent activity, the combined effect of credit card fraud drives up interest rates and costs for all customers. Generally, credit card theft has repercussions that lead to many negative factors, highlighting the necessity of continual efforts to improve cybersecurity defenses and shield people and companies from such financial and monetary loss. 
%%%%%%%%%%%%%%%%%%%%%%%%%%%%%%%%%%%%%%%%%%%%%%%%%%%%%%%%%%%%%%%%%%%%%%%%%%%%%%%%%%%
\section{Problem statement}
\label{sec:intro_prob_art}
Credit card fraud is a major financial problem because it can be tough to identify and results in significant difficulties for individuals, businesses, and financial institutions. Fraudsters attempt to avoid the system and make unauthorised purchases with another person's credit card illegally using their card information which is confidential and can be easily misused. Not only does this harm the individual whose card is taken, but it also causes significant financial losses to banks and retailers. Thus, a sophisticated system that can identify suspicious credit card transactions is needed to ensure that people's money is safe and that everyone may use credit cards without fear of criminals seeking to take advantage of them.

%%%%%%%%%%%%%%%%%%%%%%%%%%%%%%%%%%%%%%%%%%%%%%%%%%%%%%%%%%%%%%%%%%%%%%%%%%%%%%%%%%%
\section{Aims and objectives}
\label{sec:intro_aims_obj}

The main purpose of this research is to assess the performance of the user's fraud detection model utilizing various supervised machine algorithms and deep learning and artificial neural models to increase the accuracy of the model and gain a higher detection accuracy by comparing alternative approaches. A credit card fraud detection system's goal is to provide a clever and effective system that can easily identify and avoid illegal or unauthorized credit card transactions. Therefore, creating a system that can identify anomalous behaviors or fraud transactions and promptly notify individuals and financial institutions of such activities is the primary goal.



%%%%%%%%%%%%%%%%%%%%%%%%%%%%%%%%%%%%%%%%%%%%%%%%%%%%%%%%%%%%%%%%%%%%%%%%%%%%%%%%%%%
\section{Solution approach}
\label{sec:intro_sol} % label of Org section
Credit Card Fraud Detection is a vital component of guaranteeing financial security for customers and companies alike. It becomes evident that traditional machine learning techniques have not been able to provide effective solutions by analyzing the difficulties related to credit card fraud, such as the availability of public data, class imbalance in data, the changing nature of frauds, and the high number of false alarms. It is possible to create fraud detection algorithms that are more precise and efficient, though, given the recent developments in deep learning. These technologies will save financial organizations billions of dollars that are lost annually to fraud, as well as provide employment security and secure bank and credit card earnings. To improve the productivity and usefulness of its findings, the research incorporates several models. A hybrid strategy is chosen among the models examined, which include Regression-based classifiers, Decision Tree classifiers, and Boosting based models. This hybrid method optimizes using a grid-based searching algorithm and combines a tree-based model with a linear-based model. The models are optimized to maximize performance on the given dataset using methods like hyperparameter tuning. However, determining the best set of hyperparameters may be difficult and time-consuming, particularly when dealing with a wide variety of hyperparameters and intricate models like LightGBM and XGBoost. The training of the tress by gradient boost technique is optimised by the use of XG Boost which is an Extreme Gradient Boosting algorithm. It is monitored and incredibly adaptable and effective. Large data sets may be utilized for regression and classification and it has an ease of managing the missing values very well by the use of regularization techniques L1 and L2. Improvements have been demonstrated in these areas: accuracy, sensitivity, and precision by evaluating the model on certain performance metrics like recall or precision scores, F1 measures, false negatives and false positives, etc. 

Thus, by enhancing safety measures, this technology gives the banking industry a better way to identify and stop scams during online transactions.

 

\subsection{Flow Chart of Data}
\label{sec:intro_some_sub1}


Below, we see how we process the data for our Model Evaluation.

\begin{figure}[ht]
    \centering
    \includegraphics[scale=0.6]{figures/FlowChart.png}
    \caption{Figure of the Data}
    \label{fig:Plot of the Data}
\end{figure}


%%%%%%%%%%%%%%%%%%%%%%%%%%%%%%%%%%%%%%%%%%%%%%%%%%%%%%%%%%%%%%%%%%%%%%%%%%%%%%%%%%%
\section{Summary of contributions and achievements} %  use this section 
\label{sec:intro_sum_results} % label of summary of results
A major responsibility of a credit card fraud detection system is to protect individuals from fraudulent transactions which severely affects cardholders and other businesses. Its primary contribution is the detection and prevention of fraudulent activity, which helps to shield people and companies from monetary damages. These systems use the dataset provided by the user which contains relevant details like the amount of transaction, date and time of the transaction, location, history of transactions, etc. which is analyzed by using various Python libraries and then used as a training dataset for the models for trends and abnormalities using cutting-edge technology like machine learning and data analytics.  


%%%%%%%%%%%%%%%%%%%%%%%%%%%%%%%%%%%%%%%%%%%%%%%%%%%%%%%%%%%%%%%%%%%%%%%%%%%%%%%%%%%

