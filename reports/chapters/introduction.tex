\chapter{Introduction}
\label{ch:into} % This how you label a chapter and the key (e.g., ch:into) will be used to refer this chapter ``Introduction'' later in the report. 
% the key ``ch:into'' can be used with command \ref{ch:intor} to refere this Chapter.

With the rise of digital transactions, the chances of online fraud have also increased 
significantly. Credit cards have become an integral part of financial transactions which facilitates online commerce more easily and it is also convenient for customers. However, with an increase in the use of such online transactions, there is an increase in the risk of fraud which leads to huge losses for both customers and financial institutions. Credit card fraud happens when an unidentified user makes the transaction using someone else’s card information.
Detecting such fraud manually is a very challenging task due to such a high number of transactions every moment. To deal with this cybercrime, an efficient fraud detection system \citep{Dornadula-2019} should be established which uses various machine learning algorithms and models to accurately detect the fraud. In this research, a credit card fraud detection dataset  is taken from an online source Kaggle \citep{Kaggle}  where the data is cleaned and processed to extract relevant information which can be used as training data for the machine learning models to accurately detect the fraud. Some machine learning algorithms \citep{Shah-2023} that are used are logistic regression,decision tree classification, and random forest classification which can detect fraud based on the data of past transactions and can detect fraud in real-time which is a big advantage of using machine learning algorithms. After that, these models are further tuned for provided dataset  using techniques like hyperparameter tuning \citep{Dalal-2022} , which greatly enhances the model's performance. The training data used in these models are based on various factors like location, date and time of the transaction, past transactions, etc. The performance of these combined models is then evaluated using various evaluation metrics. Deployment of such fraud detection systems ensures safer transactions by improving online security.





%%%%%%%%%%%%%%%%%%%%%%%%%%%%%%%%%%%%%%%%%%%%%%%%%%%%%%%%%%%%%%%%%%%%%%%%%%%%%%%%%%%
\section{Background}
\label{sec:into_back}

With the rise of digital transactions, the risk of credit card fraud has escalated. Current 
fraud detection systems are struggling to keep pace with evolving tactics, prompting 
the need for more efficient solutions. This study utilizes machine learning algorithms such as 
Logistic Regression and Decision Trees to craft a responsive credit card fraud 
detection system. Prioritizing real-time adaptability, the project seeks to enhance the 
security of financial transactions, contributing to the ongoing battle against credit card 
fraud.

%%%%%%%%%%%%%%%%%%%%%%%%%%%%%%%%%%%%%%%%%%%%%%%%%%%%%%%%%%%%%%%%%%%%%%%%%%%%%%%%%%%
\section{Problem statement}
\label{sec:intro_prob_art}
How can machine learning algorithms be effectively employed to enhance the specific identification and timely detection of credit card fraud, ensuring measurable improvements in accuracy, efficiency, and relevance to contemporary fraud patterns?

%%%%%%%%%%%%%%%%%%%%%%%%%%%%%%%%%%%%%%%%%%%%%%%%%%%%%%%%%%%%%%%%%%%%%%%%%%%%%%%%%%%
\section{Aims and objectives}
\label{sec:intro_aims_obj}


\textbf{Aims:} The goal of this project is to use machine learning algorithms to create an effective credit card fraud detection system that will limit the scope of fraud by providing a reliable and flexible solution.




\textbf{Objectives:} First, we use machine learning methods, including Random Forest Classification, Decision Tree Classification, and Logistic Regression, to analyse historical credit card transaction data. The focus is on training the models to recognise anomalies in transaction behaviour and spotting patterns suggestive of fraudulent activity. Second, the research seeks to implement and assess the developed models in a real-time context, ensuring the system's capability to swiftly detect and mitigate potential fraud during 
live transactions. Additionally, the study aims to contribute to the existing knowledge in credit card fraud detection by evaluating and comparing the effectiveness of various 
machine learning approaches.



%%%%%%%%%%%%%%%%%%%%%%%%%%%%%%%%%%%%%%%%%%%%%%%%%%%%%%%%%%%%%%%%%%%%%%%%%%%%%%%%%%%
\section{Solution approach}
\label{sec:intro_sol} % label of Org section
In this research work, several models are merged to produce more relevant and effective results. A grid-based searching technique is then implemented and two classifiers—a tree-based model and a liner-based model—are integrated from a variety of models, including Support Vector Machine (SVM), K-means Classification, Decision Tree classifiers, and Regression-based classifiers using the training data set provided to the model as an input which is the processed data. These models are then optimized using methods like hyperparameter tweaking for the given dataset, which significantly improves the model's performance in an optimal way. However, determining the best set of hyperparameters may be difficult and time-consuming, particularly when dealing with a wide variety of hyperparameters and intricate models like LightGBM and XGBoost. The training of the trees by gradient boost technique is optimized by the use of XG Boost which is an Extreme Gradient Boosting algorithm. It is monitored and incredibly adaptable and effective. Large data sets may be utilized for regression and classification and it has an ease of managing the missing values very well by the use of regularization techniques L1 and L2. Improvements have been demonstrated in these areas: accuracy, sensitivity, and precision by evaluating the model on certain performance metrics like recall or precision scores, F1 measures, false negatives and false positives, etc

\subsection{A subsection 1}
\label{sec:intro_some_sub1}
You may or may not need subsections here. Depending on your project's needs, add two or more subsection(s). A section takes at least two subsections. 

\subsection{A subsection 2}
\label{sec:intro_some_sub2}
Depending on your project's needs, add more section(s) and subsection(s).

\subsubsection{A subsection 1 of a subsection}
\label{sec:intro_some_subsub1}
The command \textbackslash subsubsection\{\} creates a paragraph heading in \LaTeX.

\subsubsection{A subsection 2 of a subsection}
\label{sec:intro_some_subsub2}
Write your text here...

%%%%%%%%%%%%%%%%%%%%%%%%%%%%%%%%%%%%%%%%%%%%%%%%%%%%%%%%%%%%%%%%%%%%%%%%%%%%%%%%%%%
\section{Summary of contributions and achievements} %  use this section 
\label{sec:intro_sum_results} % label of summary of results
Describe clearly what you have done/created/achieved and what the major results and their implications are. 


%%%%%%%%%%%%%%%%%%%%%%%%%%%%%%%%%%%%%%%%%%%%%%%%%%%%%%%%%%%%%%%%%%%%%%%%%%%%%%%%%%%

